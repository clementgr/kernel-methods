% This version of CVPR template is provided by Ming-Ming Cheng.
% Please leave an issue if you found a bug:
% https://github.com/MCG-NKU/CVPR_Template.

\documentclass[final]{cvpr}

\usepackage{times}
\usepackage{epsfig}
\usepackage{graphicx}
\usepackage{amsmath}
\usepackage{amssymb}
\usepackage[numbers]{natbib}
\usepackage{notoccite}
\usepackage{subcaption}
\captionsetup{compatibility=false}
\usepackage{graphicx}

% Include other packages here, before hyperref.

% If you comment hyperref and then uncomment it, you should delete
% egpaper.aux before re-running latex.  (Or just hit 'q' on the first latex
% run, let it finish, and you should be clear).
\usepackage[pagebackref=true,breaklinks=true,colorlinks,bookmarks=false]{hyperref}


\def\cvprPaperID{34} % *** Enter the CVPR Paper ID here
\def\confYear{CVPR 2021}
%\setcounter{page}{4321} % For final version only


\begin{document}
	
	%%%%%%%%% TITLE
	\title{ Kernel Methods Challenge\\
		\vspace{1mm}
		\large \normalfont Predicting Whether a DNA Sequence Region is Binding Site to a Specific Transcription Factor}
	
	\author{\textbf{Clément Grisi}\\
		Team Name: clems\\
		\small \url{grisi.clement@gmail.com}
	}
	
	\maketitle
	
	\begin{abstract}
		In this report, I emphasize my work for the challenge organized as part of the Kernel Methods class. The goal of the challenge was to learn how to implement kernel-based machine learning algorithms, gain understanding about them and adapt them to structural data. To that end, teachers chose a DNA sequence classification task. Through iterations in model definition and data representation, I highlight the pros and cons of the different methods I tried, resulting in a $0.65$ classification accuracy on the academic leaderboard. My code is publicly available at \small{\url{https://github.com/clementgr/kernel-methods}}
	\end{abstract}
	
	\vspace{-3mm}
	
	\section{Introduction}
	
	Transcription factors (TFs) are regulatory proteins that bind specific sequence motifs in the genome to activate or repress transcription of target genes. Genome-wide protein-DNA binding maps can be profiled using some experimental techniques and thus all genomics can be classified into two classes for a TF of interest: bound or unbound. The challenge consists in predicting whether a DNA sequence region is binding site to a specific TF.\\
	\\
	\textbf{Dataset:} we had to work with three datasets corresponding to three different transcription factors. Each training set contained $2000$ sequences, while each testing set contained $1000$. Predictions had to be made separately for each train-test dataset pair.\\
	\\
	\textbf{Evaluation Metric:} the performance is measured using the classification accuracy. The final score is computed by taking the average of the scores from public and private leaderboards.
	
	\section{Classifiers}
	
	I implemented different classifiers for the sequence classification task.
	
	\subsection{Logistic Regression}
	
	\subsection{Support Vector Machines} 
	
	soft, using \texttt{convexopt} optimization package for solving quadratic programs.
	
	\subsection{Kernel Ridge Regression}
	
	\subsection{Kernel SVM} 
	
	soft, using \texttt{cvxpy} optimization package for solving quadratic programs
	
	\section{Working with Numeric Data}
	
	Teachers provided us with numerical feature matrices based on bag of words representation. Subsequences of length $\ell = 10$ were extracted from the DNA sequences and one-hot encoded. Then, they were clustered into $100$ clusters using Kmeans: each subsequence was assigned to a cluster $i$, thus represented by a binary vector whose coefficients are equal to $0$ except the $i$-th one, which is equal to $1$. The feature vector of each sequence simply consisted in the average of the representations of all its subsequences.
	
	\subsection{Radial-basis Function Kernel}
	
	\section{Working with Raw DNA Sequences}
	
	\subsection{k-Spectrum Kernel}
	
	\cite{spectrum}
	
	\subsection{Mismatch Kernel}
	
	\cite{mismatch}

	
	\section{Conclusion}
	
	
	{\small
		\bibliographystyle{unsrt}
		\bibliography{egbib}
	}
	
	\clearpage
	
	\begin{figure}[h!]
	\centering
	\includegraphics[width=9cm, trim=2cm 2cm 2cm 2cm, clip]{fig/spectrum/val_acc_multiple_k_dataset0.pdf}
	\includegraphics[width=9cm, trim=2cm 2cm 2cm 2cm, clip]{fig/spectrum/val_acc_multiple_k_dataset1.pdf}
	\includegraphics[width=9cm, trim=2cm 2cm 2cm 2cm, clip]{fig/spectrum/val_acc_multiple_k_dataset2.pdf}
	\caption{\centering Validation Accuracy for Different k-Spectrum Kernels as a Function of Regularization Parameter $\lambda$}
	\label{fig:acc_spectrum}
	\end{figure}
	
	\newpage
	
	\begin{figure}[h!]
		\centering
		\includegraphics[width=9cm, trim=2cm 2cm 2cm 2cm, clip]{fig/spectrum/val_acc_lambda_multiple_k_dataset0.pdf}
		\includegraphics[width=9cm, trim=2cm 2cm 2cm 2cm, clip]{fig/spectrum/val_acc_lambda_multiple_k_dataset1.pdf}
		\includegraphics[width=9cm, trim=2cm 2cm 2cm 2cm, clip]{fig/spectrum/val_acc_lambda_multiple_k_dataset2.pdf}
		\caption{\centering Optimal Regularization Parameter $\lambda$ for Different k-Spectrum Kernels}
		\label{fig:lambda_spectrum}
	\end{figure}
	
	\begin{figure}[h!]
		\centering
		\includegraphics[width=9cm, trim=2cm 2cm 2cm 2cm, clip]{fig/mismatch/val_acc_multiple_k_dataset0_mismatch.pdf}
		\includegraphics[width=9cm, trim=2cm 2cm 2cm 2cm, clip]{fig/mismatch/val_acc_multiple_k_dataset1_mismatch.pdf}
		\includegraphics[width=9cm, trim=2cm 2cm 2cm 2cm, clip]{fig/mismatch/val_acc_multiple_k_dataset2_mismatch.pdf}
		\caption{\centering Validation Accuracy for Different Mismatch Kernels as a Function of Regularization Parameter $\lambda$}
		\label{fig:acc_mismatch}
	\end{figure}
	
	\newpage
	
	\begin{figure}[h!]
		\centering
		\includegraphics[width=9cm, trim=2cm 2cm 2cm 2cm, clip]{fig/mismatch/val_acc_lambda_multiple_k_dataset0_mismatch.pdf}
		\includegraphics[width=9cm, trim=2cm 2cm 2cm 2cm, clip]{fig/mismatch/val_acc_lambda_multiple_k_dataset1_mismatch.pdf}
		\includegraphics[width=9cm, trim=2cm 2cm 2cm 2cm, clip]{fig/mismatch/val_acc_lambda_multiple_k_dataset2_mismatch.pdf}
		\caption{\centering Optimal Regularization Parameter $\lambda$ for Different Mismatch Kernels}
		\label{fig:lambda_mismatch}
	\end{figure}
		
\end{document}